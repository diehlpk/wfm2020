The comparison against experimental results is essential to validate models and discretizations in order to gain confidence in the approach’s predictability and reliability. Peridynamics and phase field methods have been utilized for fitting or validation against experimental results. When using experimental results. the question arises if the experimental data is mature enough to provide all data to set up the simulations, e.g. applied loading, and compare the experimentally measured quantity of interest with the one obtained by simulation. \\

During this workshop the phase field and peridyanmic community showcases which kind of experiments were used for validation and the difficulties phased. The experimental fracture mechanics community emphasizes the experiments they are researching on and what kind of simulations would be interesting for them to gain more understanding of experimental phenomena observed. \\

This workshop brings together experts on experimental fracture mechanics and experts in modelling and simulating crack and fractures utilizing peridyanmic and phase fields methods. As a results, it will elaborate the collaboration between the three participating communities and starts the discussion about a set of experiments used as a benchmark problem for a robust validation of phase field and peridyanmic models.




\section*{Local organizers}
\begin{itemize}
    \item Patrick Diehl
    \item Adrian Serio
\end{itemize}

\section*{Scientific committee}
\begin{itemize}
\item Pablo Seleson, Oak Ridge National Lab
\item Stuart Silling, Sandia National Lab
\item Robert Lipton, Louisiana State University
\item Serge Prudhomme, Ecole de Polytechnique Montreal
\item Patrick Diehl, Louisiana State University
\item Missing for Phase field and Experimental fracture mechanics
\end{itemize}
