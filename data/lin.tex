In this talk, I will introduce convolutional neural networks designed to predict and analyze damage patterns on a disk resulting from molecular dynamic (MD) collision simulations. The simulations under consideration are specifically designed to produce cracks on the disk and, accordingly, numerical methods which require partial derivative information, such as finite element analysis, are not applicable. These simulations can, however, be carried out using peridynamics, a nonlocal extension of classical continuum mechanics based on integral equations which overcome the difficulties in modeling deformation discontinuities. Although this nonlocal extension provides a highly accurate model for the MD simulations, the computational complexity and corresponding run times increase greatly as the simulations grow larger. We propose the use of neural network approximations to complement peridynamic simulations by providing quick estimates which maintain much of the accuracy of the full simulations while reducing simulation times by a factor of 1500. We propose two distinct convolutional neural networks: one trained to perform the forward problem of predicting the damage pattern on a disk provided the location of a colliding object’s impact, and another trained to solve the inverse problem of identifying the collision location, angle, velocity, and size given the resulting damage pattern.

