\begin{center}
\textit{Co-Authors: D. Behera and P. Roy}
\end{center} 
This study considers finite elastic deformation and damage in rubber-like materials under quasi-static loading conditions.  The peridynamic equilibrium equation is derived based on the Neo-Hookean hyperelastic model under the assumption of incompressibility.  The nonlocal deformation gradient tensor is computed in a bond-associated domain of interaction using the PD differential operator.  It is free of oscillations and spurious zero energy modes commonly observed in the PD correspondence models.  Also, it permits the direct imposition of natural and essential boundary conditions.  The validity of this approach is demonstrated through simulations of experiments concerning progressive damage and rupture in polymers undergoing large elastic deformation.\\

\noindent\textbf{References}\\
$[$1$]$ Silling SA, Epton M, Weckner O, Xu J, Askari E, Peridynamic states and constitutive
modeling. Journal of Elasticity, 88:151-184, 2007\\\newline
$[$2$]$ Madenci E, Dorduncu M, Barut A, Phan N., Weak form of peridynamics for nonlocal essential
and natural boundary conditions. Computer Methods in Applied Mechanics and Engineering.
337: 598-631, 2018\\\newline
$[$3$]$ Madenci E, Dorduncu M, Phan N, Gu X, Weak form of bond-associated non-ordinary state-
based peridynamics free of zero energy modes with uniform or non-uniform discretization,
Engineering Fracture Mechanics, 218: 106613, 2019\\\newline
$[$4$]$ Madenci E, Barut A, Doruncu M, Peridynamic differential operator for numerical analysis,
Springer, NY, 2019.  
