Nonlocality is an essential feature of the peridynamic model of solid mechanics, which treatsall internal forces as acting through finite distances. This nonlocality, which avoids the need toevaluate partial derivatives of the deformation, helps peridynamics treat singularities such asevolving cracks within its basic field equations.\\

In this talk I will offer a perspective on the significance and effect of nonlocality in theperidynamic continuum model and other theories. Nonlocality offers a mathematical tool totreat certain physical effects such as wave dispersion and attenuation in more generality than ispossible in the local theory. It enables the modeling of interesting phenomena such as solitarywaves, as well as fracture and fragmentation. It provides a natural compatibility of peridynamicswith nanoscale long-range forces. On the other hand, nonlocality is sometimes inconvenient inmacroscale simulations, for example by creating surface effects in material properties.\\

Is nonlocality real? Is it measurable in experiments or derivable from physical principles? Iwill consider some actual and hypothetical experiments that help to give insight into the properrole of nonlocality in continuum mechanics and the mechanics of defects