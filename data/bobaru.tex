\begin{center}
\textit{Co-Authors: J. Mehrmashhadi and M. Bahadori}
\end{center} 
We evaluate the performance of three different computational models (an in-house meshfree implementation of peridynamics [1], a discontinuous-Galerkin implementation of peridynamics available in LS-DYNA [2], and a phase-field model implemented in COMSOL, similar to the implementation in [3]) on dynamic brittle fracture in soda-lime glass induced by impact. The evaluation is made against some recent experimental data from [4]. The three models show different crack branching patterns, with the angle of branching being the most significant difference. The fracture pattern predicted by the peridynamic model using the meshfree discretization matches the experimental observations, including some very fine details: small twists in the crack paths as the two crack branches approach the far-right boundary of the sample. The results from the LS-DYNA’s Discontinuous Galerkin implementation of peridynamics show spurious/secondary crack branching events, likely due to the way damage growth is implemented in this model. With the phase-field model, the crack branching angle is significantly smaller than in reality and damage continues to “diffuse” into the body after the crack has passed a certain location. A method to prevent this continuous damage expansion in phase-field models of dynamic fracture might be able to improve the phase-field-based results.\\

\noindent\textbf{References}\\
$[$1$]$ F. Bobaru, and G. Zhang, “Why do cracks branch? A peridynamic investigation of dynamic brittle fracture, Int. J. Fracture, 196(1-2), 59-98, 2015. \\\newline
$[$2$]$ B. Ren, C. T. Wu, and E. Askari, A 3D discontinuous Galerkin finite element method with the bond-based peridynamics model for dynamic brittle failure analysis, Int. J. Impact. Eng., 99, 14-25, 2017.\\\newline 
$[$3$]$ S. W. Zhou, T. Rabczuk, and X. Y. Zhuang, “Phase field modeling of quasi-static and dynamic crack propagation: COMSOL implementation and case studies,” Adv. Eng. Softw., 122, 31-49, 2018. \\\newline
$[$4$]$ B. M. Sundaram, and H. V. Tippur, Dynamic fracture of soda-lime glass: A full-field optical investigation of crack initiation, propagation and branching, J. Mech. Phys. Solids, 120, 132-153, 2018
