\begin{center}
\textit{Co-Authors: Aditya Kumar, Blaise Bourdin, and Gilles A. Francfort}
\end{center} 
Twenty years in since their introduction [1], it is now plain that the regularized formulations dubbed as phase-field of the variational theory of brittle fracture of Francfort and Marigo [2] provide a powerful macroscopic theory to describe and predict the propagation of cracks in linearelastic brittle materials under arbitrary quasistatic loading conditions. Over the past ten years, the ability of the phase-field approach to also possibly describe and predict crack nucleation has been under intense investigation. The first of two objectives of this talk is to establish that the existing phase-field approach to fracture at large — irrespectively of its particular version — is fundamentally incomplete to model crack nucleation. This is because the approach is purely energetic and cannot account for one essential ingredient that is not energetic but rather stress-based: the strength of the material.  The second objective is to introduce an amendment that renders a phase-field theory capable of modeling crack nucleation in general, be it from large pre-existing cracks, small pre-existing cracks, smooth and non-smooth boundary points, or within the bulk of structures subjected to arbitrary quasistatic loadings. Following Kumar, Francfort, and Lopez-Pamies [3], the central idea is to implicitly account for the presence of the inherent microscopic defects in the material — whose defining macroscopic manifestation is precisely the strength of the material — through the addition of an external driving force in the equation governing the evolution of the phasefield. To illustrate the descriptive and predictive capabilities of the proposed theory, the last part of the talk will be devoted to present sample simulations of experiments spanning the full range of fracture nucleation settings.\\

\noindent\textbf{References}\\
$[$1$]$ Bourdin, B., Francfort, G.A., Marigo, J.J., Numerical experiments in revisited brittle fracture, Journal of the Mechanics and Physics of Solids, 48, 797–826, 2000.\\\newline
$[$2$]$ Francfort, G.A., Marigo, J.J., Revisiting brittle fracture as an energy minimization problem, Journal of the Mechanics and Physics of Solids, 46, 1319–1342, 1998.\\\newline
$[$3$]$ Kumar, A., Francfort, G.A., Lopez-Pamies, O., Fracture and healing of elastomers: Aphase-transition theory and numerical implementation, Journal of the Mechanics and Physics of Solids, 112, 523–551, 2018.



