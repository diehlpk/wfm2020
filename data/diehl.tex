\begin{center}
\textit{Co-Authors: Serge Prudhomme and Martin Lévesque}
\end{center} 
Peridynamics (PD), a non-local generalization of classical continuum mechanics (CCM) allowing for discontinuous displacement fields, provides an attractive framework for the modeling and simulation of fracture mechanics applications. However, PD introduces new model parameters, such as the so-called horizon parameter. The length scale of the horizon is a priori unknown and need to be identified. Moreover, the treatment of the boundary conditions is also problematic due to the non-local nature of PD models. It has thus become crucial to calibrate the new PD parameters and assess the model adequacy based on experimental observations. The objective of the present paper is to review and catalog available experimental setups that have been used to date for the calibration and validation of peridynamics. We have identified and analyzed a total of 39 publications that compare PD-based simulation results with experimental data. In particular, we have systematically reported, whenever possible, either the relative error or the R-squared coefficient. The best correlations were obtained in the case of experiments involving aluminum and steel materials. Experiments based on imaging techniques were also considered. However, images provide large amounts of information and their comparison with simulations is in that case far from trivial. A total of six publications have been identified and summarized that introduce numerical techniques for extracting additional attributes from peridynamics simulations in order to facilitate the comparison against image-based data.\\

\noindent\textbf{References}\\
$[$1$]$ Diehl, P., Prudhomme, S., \& Lévesque, M. (2019). A review of benchmark experiments for the validation of peridynamics models. Journal of Peridynamics and Nonlocal Modeling, 1(1), 14-35.
