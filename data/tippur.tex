\begin{center}
\textit{Co-Author: Sivareddy Dondeti}
\end{center} 
The dynamic fracture of high-stiffness and low-toughness materials such as soda-lime silicate glass (SLSG) involves crack initiation and growth prior to branching, underlying mechanics of which is not yet understood. Addressing this issue using full-field optical techniques have faced numerous spatio-temporal challenges since crack speeds in this material reach excess of 1500 m/s and are accompanied by highly localized sub-micron scale deformations. Recent work by the authors have shown that transmission-mode Digital Gradient Sensing method [1, 2] is capable of overcoming many of these challenges to visualize crack-tip fields in the whole field and quantify fracture parameters associated with each of the phases of crack growth in SLG plates. In this work, time-resolved stress gradient and stress measurements in SLSG plates of two different geometries subjected to dynamic wedge-loading are performed. The LEFM-based precursors extracted from optical measurements leading up to single or sequential/cascading branch formations in SLSG are reported. The identification of precursors are based on crack velocity, stress intensity factors, higher order coefficients of the asymptotic crack tip fields and non-dimensional parameters based on a combination of these. Fracture surface roughness and its features are also separately quantified using high resolution post-mortem examination and corroboration with optically measured quantities.\\

\noindent\textbf{References}\\
$[$1$]$ B.M. Sundaram and H.V. Tippur, Full-field measurement of contact-point and crack-tip deformations in soda-lime glass. Part-II: Stress wave loading, International Journal of Applied Glass Science, 9, 123-136, 2018.\\\newline 
$[$2$]$ B.M. Sundaram and H.V. Tippur, Dynamic fracture of soda-lime glass: A quantitative full-field optical investigation of crack initiation, propagation and branching, Journal of the Mechanics and Physics of Solids, 120, 132-153, 2018. 
