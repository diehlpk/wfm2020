One hundred years after Griffith’s landmark paper on the theory of rupture, the problem of fracture remains of significant interest. While much success has been achieved in the reliability assessment of structures with cracks, much remains to be done in terms of predictive assessment of failure. In this talk, I will focus on the crucial role played by experiments in all aspects of the development of the theory of rupture, providing physical insight and detailed measurements to challenge and to stimulate modeling and simulation methods. Specifically, I will describe a selection of fracture mechanics experiments, encompassing quasi-static and dynamic fracture as well as nucleation of damage and cracks in brittle and ductile materials. Leveraging on the advances in experimental measurement, I will attempt to pose challenges for proper validation of modeling and simulation methods.
